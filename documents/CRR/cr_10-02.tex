\documentclass{scrartcl}
\usepackage[margin=0.7in]{geometry} %réduire marges

\setkomafont{disposition}{\normalfont\bfseries}

%french
\usepackage[utf8]{inputenc}
\usepackage[T1]{fontenc}
\usepackage{lmodern}

\begin{document}
\pagenumbering{gobble}%pas num page

\title{Compte rendu de la réunion du 10/02/15}
\subtitle{\textit{Prochaine réunion 18 février, 17h15}\vspace{-5ex}}
\date{}
\maketitle

La présentation a été faite par Adrien.
Nous avons obtenu un terminal Apple pour les tests.

\section{Bibliothèque de son présentée}
\begin{itemize}
  \item http://www.cocoawithlove.com/2010/10/ios-tone-generator-introduction-to.html

  \item Ne correspond pas exactement à l'attente du client, car le son généré par l'application est monotone, et il faudrait un son de guitare

  \item Peut être utilisée dans un premier temps pour tester l'interface graphique de l'application avant de trouver une meilleure bibliothèque

\end{itemize}


\section{Problèmes}
\begin{itemize}
  \item Impossible de créer un repository sur les machines de la fac à cause du proxy : demander à la PPTI
\end{itemize}

\section{Conseils pour la présentation}
\begin{itemize}
  \item Faire des rappelles du contexte : rappel du sujet de projet, résumé des réunions précédentes
  \item Numérotation des slides
\end{itemize}


\section{Interface graphique de l'application}
\begin{itemize}

  \item Le manche de guitare se « slide » de gauche à droite car seul un nombre limité de frettes sont visibles à l'écran (le nombre de frettes dépendra donc de la taille de l'écran). Pour atteindre des notes plus aigus, l'utilisateur doit « slider » le manche (utilisation possible d'une scrollview)
  
  \item La barre de menu n'est pas une priorité. En revanche, la barre de score est importante car elle a une certaine proportion de vert (pour les bonnes réponses) et de rouge (pour les mauvaises réponses) indiquant le score de l'utilisateur.

  \item Deux possibilités de jeu :
	\begin{itemize}
		\item Une portée et un manche sur lequel on place les notes qu'on lit sur la portée
		\item Un manche et un clavier sur lequel on choisit les notes qu'on lit sur le manche
	\end{itemize}
  \item Pour plus de clarté :
    \begin{itemize}
    \item Signaler à l'utilisateur que la note qu'il lit sur la portée se situe à une octave au dessus ou en dessous 
    \item Indiquer le numéro de la frette sur les cases du manche de guitare qui ont des indicateurs
	\end{itemize}
\end{itemize}

\section{Objectifs pour la prochaine réunion}
\begin{itemize}
  \item Avoir un exemple à montrer 
  \item Commencer la rédaction du cahier des charges
\end{itemize}

\end{document}
