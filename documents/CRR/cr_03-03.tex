\documentclass{scrartcl}
\usepackage[margin=0.7in]{geometry} %réduire marges

\setkomafont{disposition}{\normalfont\bfseries}

%french
\usepackage[utf8]{inputenc}
\usepackage[T1]{fontenc}
\usepackage{lmodern}
\usepackage[babel=true]{csquotes}
\usepackage[french]{babel} %en Français

\begin{document}
\pagenumbering{gobble}%pas num page

\title{Compte rendu de réunion du 03/03/15}
\subtitle{Prochaine réunion : 18 mars, 17h.\vspace{-5ex}}
\date{}
\maketitle

\textit{La présentation a été faite par Alexandra.}

\section{Conseils pour la présentation}
\begin{itemize}
	\item  Remanier la slide d'intro, son niveau de détail, la standardiser
	\item  Ne pas varier les niveaux d'abstraction
	\item  Bien distinguer les perspectives d'améliorations des directives du client
	\item  Lister sur la slide de fin l'ensemble des questions diséminées dans la présentation
\end{itemize}

\section{Le clavier, expérimentations}
\begin{itemize}
	\item  Essayer les segmented control
	\item  Réduire la taille des touches
	\item  Contrôler le rendu sous différents terminaux
	\item  Sous réserve de l'en prévenir assez tôt, M. Kordon s'est proposé d'installer nos maquettes sur différents terminaux
\end{itemize}

\section{À produire}
\begin{itemize}
	\item  Donner les liens, catégories, conclusions résultant de l'investigation des "concurents" du iBalezator
	\item  Rédiger les scénarios qui seront utiles lors de la phase de recettage
	\item  Fournir des maquettes des différentes expérmentations sur l'agencement du clavier
	\item  Fournir une maquette du changement de mode par "glissement"
	\item  Un diagramme de classe sur l'architecture logicielle de l'appli
\end{itemize}
\end{document}
