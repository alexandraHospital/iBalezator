\documentclass{scrartcl}
\usepackage[margin=0.7in]{geometry} %réduire marges

\setkomafont{disposition}{\normalfont\bfseries}

%french
\usepackage[utf8]{inputenc}
\usepackage[T1]{fontenc}
\usepackage{lmodern}
\usepackage[babel=true]{csquotes}
\usepackage[french]{babel} %en Français

\begin{document}
\pagenumbering{gobble}%pas num page

\title{Compte rendu de réunion du 16/14/15}
\subtitle{Prochaine réunion : jeudi 30 avril, 18h.\vspace{-5ex}}
\date{}
\maketitle

\textit{La présentation a été faite par Alexandra.}

\section{Consignes pour la soutenance}
\begin{itemize}
	\item Expliquer le projet/sujet de zéro
	\item Présenter ce qui a été fait, démo,  les problèmes rencontrés, conclure
	\item Parler des évolutions possibles
	\item Nous nous adresserons à des informaticiens mais éviter les détails trop techniques
\end{itemize}

\section{Conseils pour l'implémentation}
\begin{itemize}
	\item Utiliser les NSTimer pour la temporisation
	\item Quand le timer est armé, il faut ignorer les intéractions de l'utilisateur
	\item Penser à faire la vue d'aide
	\item Il est possible de placer une vue par rapport à son centre
	\item Enlever toutes les constantes non-déclarées (\og Magic Numbers \fg{})
	\item Une vue peut être rafraîchie grâce à \texttt{setNeedDisplay()}
	\item En mode Portée/Manche, un problème se produit quand on sélectionne la 6\up{ème} corde, 1\up{ère} frette
	\item Mettre en place l'icône de l'application
	\item Réfléchir à la persistance des données
\end{itemize}

\end{document}
