\documentclass{scrartcl}
\usepackage[margin=0.7in]{geometry} %réduire marges

\setkomafont{disposition}{\normalfont\bfseries}

%french
\usepackage[utf8]{inputenc}
\usepackage[T1]{fontenc}
\usepackage{lmodern}
\usepackage[babel=true]{csquotes}
\usepackage[french]{babel} %en Français

\begin{document}
\pagenumbering{gobble}%pas num page

\title{Compte rendu de réunion du 26/03/15}
\subtitle{Prochaine réunion : jeudi 2 avril, 18h.\vspace{-5ex}}
\date{}
\maketitle

\textit{La présentation a été faite par Alexandra.}

\section{Conseils pour la présentation}
\begin{itemize}
	\item Commencer la présentation par la version IOS du Balézator
	\item Ne pas oublier de définir le terme "mode de jeu"
	\item Les slides de présentation ne devraient pas excéder 2 minutes
	\item Les slides ne doivent pas être trop chargées en texte
	\item Tâcher de faire une présentiation énergique, qui respecte le temps imparti
	\item Penser à :
		\begin{itemize}
			\item une conclusion, 
			\item une slide de fin qui résume les questions soulevées,
			\item une slide qui résume ce qui a pu être fait par rapport aux objectifs,
			\item rappeler où l'on en est, où l'on va.
		\end{itemize}
\end{itemize}

\section{Remarques sur l'existant}
\begin{itemize}
	\item Les maquettes ont été approuvées par Thomas
	\item Penser aux init de convenience pour attacher un contrôleur à une vue
	\item La barre d'info doit :
		\begin{itemize}
			\item Être de taille fixe (non-proportionnelle à la taille de l'écran donc)
			\item Réspecter les dimensions décrites dans les lignes de conduites graphiques dictées par Apple
		\end{itemize}
\end{itemize}

\section{Spécifications Techniques de Réalisation (STR)}
Les STR doivent :
\begin{itemize}
	\item Donner l'architecture fonctionnelle de l'appli
	\item Décrire l'utilité de chaque classe
	\item Montrer l'intéraction entre classes
	\item Préciser où sont stockées les données
	\item Présenter l'algorithme des traitements complexes
\end{itemize}

\section{Planning prévisionnel des réunions à venir}
\begin{itemize}
	\item Jeudi 16 avril
	\item Jeudi 30 avril (18h00)
	\item Mardi 12 mai (11h30) Avis final donné par Thomas sur l'application livrée. 
\end{itemize}
\end{document}
