\documentclass{scrartcl}
\usepackage[margin=0.7in]{geometry} %réduire marges

\setkomafont{disposition}{\normalfont\bfseries}

%french
\usepackage[utf8]{inputenc}
\usepackage[T1]{fontenc}
\usepackage{lmodern}
\usepackage[babel=true]{csquotes}
\usepackage[french]{babel} %en Français

\begin{document}
\pagenumbering{gobble}%pas num page

\title{Compte rendu de la réunion du 24/02/15}
\subtitle{\textit{Prochaine réunion le 3 mars, 16h}\vspace{-5ex}}
\date{}
\maketitle

\textit{La présentation a été faite par Adrien.}

\section{Modèle MVC}
\begin{itemize}
 \item Compréhension du modèle MVC en Swift
 \item Lien possible entre le Model et la View
 \item Controller global qui \enquote{contrôle tout}
\end{itemize}


\section{Cahier des charges}
\begin{itemize}
 \item Ajouter des scenarios types
 \item Ajouter des maquettes
 \item Mentionner la temporisation du son, la durée de validation ou du refus de la note
\end{itemize}

\section{Pistes pour iBalezator}
\begin{itemize}
 \item Utiliser le MIDI pour la numérotation des notes
 \item OpenAL
 \item Wave guide, boucle de retard
\end{itemize}
\end{document}
