\documentclass{scrartcl}
\usepackage[margin=0.7in]{geometry} %réduire marges

\setkomafont{disposition}{\normalfont\bfseries}

%french
\usepackage[utf8]{inputenc}
\usepackage[T1]{fontenc}
\usepackage{lmodern}

\begin{document}
\pagenumbering{gobble}%pas num page

\title{Compte rendu de la réunion du 18/02/15}
\subtitle{\textit{Prochaine réunion 24 mars, 16h}\vspace{-5ex}}
\date{}
\maketitle

La présentation a été faite par Alexandra.

\section{Points abordés}
\begin{itemize}
\item Les notes sur la portée doivent correspondre à l'affichage courant du manche
\item Possibilté de déplacer le manche par crans (par frette)
\item Abandon du calcul de la position des frettes, on utilise l'image du client
\item Il est demandé de fournir un codage des notes au format MIDI
\item Utiliser le MIDI, standard dans le milieu musical, authorisera une évolution possible de l'application
\item Utilisation également de la norme MIDI pour faire correspondre le nom, la fréquence, et l'identifiant de d'une note : http://newt.phys.unsw.edu.au/jw/graphics/notes.GIF
\item Le but n'est pas de coller précisement à la version HTML/JavaScript.
\item Commiter les codes-source déjà codés
\end{itemize}


\section{Cahier des charges}
\begin{itemize}
\item Une architecture réfléchie doit être présentée dans le cahier des charges.
\item Ne pas hésiter à détailler le découpage des vues.
\item Utiliser le design pattern MVC
\item Contient des maquettes de principe pour pouvoir de se mettre d'accord sur le livrable
\end{itemize}

\section{Objectifs pour la prochaine réunion}
\begin{itemize}
\item Premier jet du cahier des charges
\item Présentation d'un schéma d'architecture de code (UML)
\item Maquette du deuxième mode de jeu : Manche-Clavier
\item Conversion tap vers note vers fréquence vers son
\item Parler de la division des tâches
\end{itemize}

\end{document}
