\documentclass{scrartcl}
\usepackage[margin=0.7in]{geometry} %réduire marges

\setkomafont{disposition}{\normalfont\bfseries}

%french
\usepackage[utf8]{inputenc}
\usepackage[T1]{fontenc}
\usepackage{lmodern}
\usepackage[babel=true]{csquotes}
\usepackage[french]{babel} %en Français

\begin{document}
\pagenumbering{gobble}%pas num page

\title{Compte rendu de réunion du 02/04/15}
\subtitle{Prochaine réunion : jeudi 16 avril, 18h.\vspace{-5ex}}
\date{}
\maketitle

\textit{La présentation a été faite par Adrien.}

\section{Conseils pour la présentation}
\begin{itemize}
	\item Se présenter, présenter son équipe
	\item La partie présentation du projet doit durer 1min30 - 2min
	\item Donner des titres positifs, \enquote{amélioration du produit} au lieu de \enquote{résolution de problèmes} par exemple

\end{itemize}


\section{STR}
\begin{itemize}
	\item Ajouter les cardinalités dans le diagramme de classe
	\item Relier la partie audio au code de la guitare sur le diagramme de classe
	\item Justifier ses choix d'architecture d'application, faire un paragraphe explicatif en dessous du diagramme de classe
	\item Décrire la lecture des notes sur la portée de la même façon que la lecture des notes sur le manche de la guitare
\end{itemize}

\section{Conseils pour l'implémentation}
\begin{itemize}
	\item Trouver une icône, penser aux détails
	\item Commencer par des petites fonctions pour tendre vers les fonctionnalités plus compliquées
\end{itemize}

\end{document}
